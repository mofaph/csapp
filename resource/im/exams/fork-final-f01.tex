%%COMMENT:8:16:Process control and signals:
\begin{problem}{16}

This problem tests your understanding of exceptional control flow in C
programs. Assume we are running code on a Unix machine. The following
problems all concern the value of the variable {\tt counter}.

\end{problem}

\subsection*{Part I (6 points)}

\begin{ccode}
\begin{alltt}
int counter = 0;

int  main()
\verb:{:
    int i;
    
    for (i = 0; i < 2; i ++)\verb:{:
        fork();
        counter ++;
        printf("counter = %d\verb:\n:", counter);
    \verb:}:
    
    printf("counter = %d\verb:\n:", counter);
    return 0;
\verb:}:
\end{alltt}
\end{ccode}

\vspace{0.30in}

A. How many times would the value of {\tt counter} be printed: \verb| ____________| 

\vspace{0.25in}

B. What is the value of {\tt counter} printed in the first line?  \verb| ____________| 

\vspace{0.25in}

C. What is the value of {\tt counter} printed in the last line? \verb| ____________| 

\newpage
\subsection*{Part II (6 points)}
\begin{ccode}
\begin{alltt}
pid_t pid;
int counter = 0;

void handler1(int sig) 
\verb:{:
    counter ++;
    printf("counter = %d\verb:\n:", counter);
    fflush(stdout);   /* Flushes the printed string to stdout */
    kill(pid, SIGUSR1);
\verb:}:

void handler2(int sig) 
\verb:{:
    counter += 3;
    printf("counter = %d\verb:\n:", counter);
    exit(0);
\verb:}:

main() \verb:{:
    signal(SIGUSR1, handler1);
    if ((pid = fork()) == 0) \verb:{:
        signal(SIGUSR1, handler2);
        kill(getppid(), SIGUSR1);
        while(1) \verb:{:\verb:}:;
    \verb:}:
    else \verb:{:
        pid_t p; int status;
        if ((p = wait(&status)) > 0) \verb:{:
            counter += 2;
            printf("counter = %d\verb:\n:", counter);
        \verb:}:
    \verb:}:
\verb:}:
\end{alltt}
\end{ccode}

\vspace{0.25in}

What is the output of this program?


\newpage
\subsection*{Part III (4 points)}
\begin{ccode}
\begin{alltt}
int counter = 0;

void handler(int sig)
\verb:{:
    counter ++;
\verb:}:


int  main()
\verb:{:
    int i;
    
    signal(SIGCHLD, handler);
    
    for (i = 0; i < 5; i ++)\verb:{:
        if (fork() == 0)\verb:{:
            exit(0);
        \verb:}:
    \verb:}:

    /* wait for all children to die */
    while (wait(NULL) != -1);
    
    printf("counter = %d\verb:\n:", counter);
    return 0;
\verb:}:
\end{alltt}
\end{ccode}

\vspace{0.30in}
A. Does the program output the same value of {\tt counter} every time we run it? \verb|  | {\tt Yes} \verb|  | {\tt No}

\vspace{0.25in}
B. If the answer to A is {\tt Yes}, indicate the value of the {\tt
counter} variable. Otherwise, list all possible values of the {\tt counter} variable. 

\vspace{0.15in}
Answer:  {\tt counter} =  \verb| __________________| 


%Answer:
%Part I:   A. 10 
%	   B. 1  
%	   C. 2
%Part II:  counter = 1
%	   counter = 3
%          counter = 3
%Part III: A. No
%          B. 1,2,3,4,5
%
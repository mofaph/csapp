%%COMMENT:3:10:Decompiling a C ``for'' loop:
\begin{problem}{10}
Condider the following assembly code for a C {\tt for} loop:

\begin{verbatim}
loop:
        pushl %ebp
        movl %esp,%ebp
        movl 0x8(%ebp),%edx
        movl %edx,%eax
        addl 0xc(%ebp),%eax
        leal 0xffffffff(%eax),%ecx
        cmpl %ecx,%edx
        jae .L4
.L6:
        movb (%edx),%al
        xorb (%ecx),%al
        movb %al,(%edx)
        xorb (%ecx),%al
        movb %al,(%ecx)
        xorb %al,(%edx)
        incl %edx
        decl %ecx
        cmpl %ecx,%edx
        jb .L6
.L4:
        movl %ebp,%esp
        popl %ebp
        ret
\end{verbatim}

Based on the assembly code above, fill in the blanks below in its
corresponding C source code.  (Note: you may only use the symbolic
variables {\tt h}, {\tt t} and {\tt len} in your expressions below
--- {\em do not use register names.})

\begin{verbatim}

void loop(char *h, int len)
{
   char *t;

   for (_____________; ___________; h++,t--) {

      _______________;

      _______________;

      _______________;
   }

   return;
}

\end{verbatim}

% SOLUTION:
% void loop(char * h, int len)
%{
%	 char *t;
%        for(t=h+len-1;h<t;h++,t--){
%                *h=*h^*t;
%                *t=*h^*t;
%                *h=*h^*t;
%        }
%}

\end{problem}

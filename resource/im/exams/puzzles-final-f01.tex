%%COMMENT:2:20:Integer and floating point puzzles:
\begin{problem}{20}
We are running programs on a machine with the following characteristics:
\begin{itemize}
\item
Values of type {\tt int} are 32 bits. They are represented in two's complement, and they are right shifted arithmetically.
Values of type {\tt unsigned} are 32 bits.

\item Values of type {\tt float} are represented using the 32-bit IEEE floating point format, while values of type {\tt double} use the 64-bit IEEE floating point format.
\end{itemize}

We generate arbitrary values {\tt x}, {\tt y}, and {\tt z}, and
convert them to other forms as follows:

\begin{ccode}
\begin{alltt}
/* Create some arbitrary values */
int x = random();
int y = random();
int z = random();
/* Convert to other forms */
unsigned ux = (unsigned) x;
unsigned uy = (unsigned) y;
double   dx = (double) x;
double   dy = (double) y;
double   dz = (double) z;
\end{alltt}
\end{ccode}

For each of the following C expressions, you are to indicate whether or not the expression {\it always} yields 1.  If so, circle ``Y''.  If not, circle ``N''.
You will be graded on each problem as follows:
\begin{itemize}
\item If you circle no value, you get 0 points.
\item If you circle the right value, you get 2 points.
\item If you circle the wrong value, you get $-1$ points 
(so don't just guess wildly).
\end{itemize}

\begin{center}
\renewcommand{\arraystretch}{1.5}
\begin{tabular}{|l|ll|}
\hline
Expression & \multicolumn{2}{|c|}{Always True?} \\
\hline
\verb@(x<y) == (-x>-y)@&Y&N\\\hline
\verb@((x+y)<<4) + y-x == 17*y+15*x@&Y&N\\\hline
\verb@~x+~y+1 == ~(x+y)@&Y&N\\\hline
\verb@ux-uy == -(y-x)@&Y&N\\\hline
\verb@(x >= 0) || (x < ux)@&Y&N\\\hline
\verb@((x >> 1) << 1) <= x@&Y&N\\\hline
\verb@(double)(float) x == (double) x@&Y&N\\\hline
\verb@dx + dy == (double) (y+x)@&Y&N\\\hline
\verb@dx + dy + dz == dz + dy + dx@&Y&N\\\hline
\verb@dx * dy * dz == dz * dy * dx@&Y&N\\\hline
\hline
\end{tabular}
\end{center}

\comment{
Answers:

Expression				Always True?

1. (x<y) == (-x>-y)			No: Let x = Tmin, y = 0
2. ((x+y)<<4) + y-x == 17*y+15*x	Yes: Associative, commutative, distributes
3. ~x+~y+1 == ~(x+y)			Yes: (-x-1)+(-y-1)+1 == -(x+y)-1				
4. ux-uy == -(y-x)			Yes:	
5. (x >= 0) || (x < ux)			No: x = -1.  Comparison x < ux is never true.
6. ((x >> 1) << 1) <= x			Yes: x>>1 rounds toward minus infinity.
7. (double)(float) x == (double) x	No: Try x = Tmax.	
8. dx + dy == (double) (y+x)		No: Try x=y=Tmin.
9. dx + dy + dz == dz + dy + dx		Yes: Within range of exact representation by double's.
10. dx * dy * dz == dz * dy * dx	No: Try dx=Tmax, dy=Tmax-1, dz=Tmax-2 	
}

\end{problem}


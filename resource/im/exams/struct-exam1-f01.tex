%%COMMENT:2:9:Structure layout and access:
\begin{problem}{9}
Consider the following C declarations:

\begin{ccode}
\begin{alltt}
typedef struct \verb:{:
    short code;
    long start;
    char raw[3];
    double data;
\verb:}: OldSensorData;

typedef struct \verb:{:
    short code;
    short start;
    char raw[5];
    short sense;
    short ext;
    double data;
\verb:}: NewSensorData;
\end{alltt}
\end{ccode}

\end{problem}

\begin{subproblem}

\item
Using the templates below (allowing a maximum of 24 bytes), indicate
the allocation of data for structs of type {\tt OldSensorData} {\tt
NewSensorData}.  Mark off and label the areas for each individual
element (arrays may be labeled as a single element).  {\bf Cross hatch the
parts that are allocated, but not used (to satisfy alignment).}

Assume the Linux alignment rules discussed in class.  {\bf Clearly
indicate the right hand boundary of the data structure with a vertical
line.}

{\tt OldSensorData:}

{\small
\begin{verbatim}
  0  1  2  3  4  5  6  7  8  9 10 11 12 13 14 15 16 17 18 19 20 21 22 23 
+--+--+--+--+--+--+--+--+--+--+--+--+--+--+--+--+--+--+--+--+--+--+--+--+
|                                                                       |
+--+--+--+--+--+--+--+--+--+--+--+--+--+--+--+--+--+--+--+--+--+--+--+--+
\end{verbatim}
}

{\tt NewSensorData:}

{\small
\begin{verbatim}
  0  1  2  3  4  5  6  7  8  9 10 11 12 13 14 15 16 17 18 19 20 21 22 23 
+--+--+--+--+--+--+--+--+--+--+--+--+--+--+--+--+--+--+--+--+--+--+--+--+
|                                                                       |
+--+--+--+--+--+--+--+--+--+--+--+--+--+--+--+--+--+--+--+--+--+--+--+--+
\end{verbatim}
}


\answer{


{\tt OldSensorData}

{\small
\begin{verbatim}
  0  1  2  3  4  5  6  7  8  9 10 11 12 13 14 15 16 17 18 19 20 21 22 23 
+--+--+--+--+--+--+--+--+--+--+--+--+--+--+--+--+--+--+--+--+--+--+--+--+
|code |XXXXX|   start   |  raw   |XX|         data          |           |
+--+--+--+--+--+--+--+--+--+--+--+--+--+--+--+--+--+--+--+--+--+--+--+--+
\end{verbatim}
}



{\tt NewSensorData}

{\small
\begin{verbatim} 
  0  1  2  3  4  5  6  7  8  9 10 11 12 13 14 15 16 17 18 19 20 21 22 23 
+--+--+--+--+--+--+--+--+--+--+--+--+--+--+--+--+--+--+--+--+--+--+--+--+
|code |start|     raw      |XX|sense| ext |XXXXX|        data           |
+--+--+--+--+--+--+--+--+--+--+--+--+--+--+--+--+--+--+--+--+--+--+--+--+
\end{verbatim}
}

} % end of answer


\newpage

\item
Now consider the following C code fragment:

\begin{ccode}
\begin{alltt}
void foo(OldSensorData *oldData)
\verb:{:
    NewSensorData *newData;

    /* this zeros out all the space allocated for oldData */
    bzero((void *)oldData, sizeof(oldData));

    oldData->code = 0x104f;
    oldData->start = 0x80501ab8;
    oldData->raw[0] = 0xe1;
    oldData->raw[1] = 0xe2;
    oldData->raw[2] = 0x8f;
    oldData->raw[-5] = 0xff;
    oldData->data = 1.5;

    newData = (NewSensorData *) oldData;

    ...
\end{alltt}
\end{ccode}

Once this code has run, we begin to access the elements of newData.  Below,
give the value of each element of {\tt newData} that is listed.  Assume that 
this code is run on a Little-Endian machine such as a Linux/x86 machine.
You must give your answer in hexadecimal format.
{\bf Be careful about byte ordering!}.

\begin{subproblem}
\item \verb:newData->start  = 0x________________: \medskip \answer{0xff00}
\item \verb:newData->raw[0] = 0x________________: \medskip \answer{0xb8}
\item \verb:newData->raw[2] = 0x________________: \medskip \answer{0x50}
\item \verb:newData->raw[4] = 0x________________: \medskip \answer{0xe1}
\item \verb:newData->sense  = 0x________________: \medskip \answer{0x008f}
\end{subproblem}

\end{subproblem}



        





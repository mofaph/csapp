%%COMMENT:6:10:High-level cache miss rate analysis:
\begin{problem}{10}
\end{problem}
After a stressful semester you suddenly realize that
you haven't bought a single christmas present yet.
Fortunately, you see that one of the big electronic
stores has CD's on sale.
You don't have much time to decide which CD will make
the best presents for which friend,
so you decide to automatize the decision process.
For that, you use a database containing an entry for each
of your friends.
It is is implemented
as an $8\times8$ matrix using a data structure {\tt person}. 
You add to this data structure a field for each CD that you consider:\\

{\small
\begin{verbatim}
struct person{

  char name[16];

  int age;
  int male;

  short nsync;
  short britney_spears;
  short dolly_parton;
  short garth_brooks;
}

struct person db[8][8];
register int i, j;

\end{verbatim}
}

\newpage
\subsection*{Part 1}
After thinking for a while you come up with the following 
smart routine that finds the ideal present for everyone.

{\small
\begin{verbatim}
void generate_presents(){

  for (j=0; j<8; j++){
      for (i=0; i<8; i++) {
          db[i][j].nsync=0;
          db[i][j].britney_spears=0;
          db[i][j].garth_brooks=0;
          db[i][j].dolly_parton=0;
      }
   }  

  for (j=0; j<8; j++){
      for (i=0; i<8; i++) {        

          if(db[i][j].age < 30){
             if(db[i][j].male)
                   db[i][j].britney_spears = 1;
             else  db[i][j].nsync = 1;
          }
          else{
             if(db[i][j].male) 
                   db[i][j].dolly_parton =1;
             else  db[i][j].garth_brooks = 1;
          }
      }
   }      
  
}
\end{verbatim}
}


Of course, runtime is important in this time-critical application,
so you decide to analyze the cache performance of your  routine.
You assume that
\begin{itemize}
\item your machine has a 512-byte direct-mapped data cache with 64 byte 
blocks. 
\item \verb|db| begins at memory address 0
\item The cache is initially empty.
\item The only memory accesses are to the entries of the array \verb|db|.
Variables \verb|i|, and  \verb|j| are stored in registers.
\end{itemize} 

\vspace{\baselineskip}
Answer the following questions:

A. What is the total number of read and write accesses? \verb|_______|. 

% Answer: 4*8*8 + 3*8*8 = 448 

B. What is the total number of read and write accesses that 
miss in the cache? \verb|_______| .

% Answer: 1*8*8 + 1*8*8 = 128

C. So the fraction of all accesses that miss in the cache is: \verb|_______|.

% Answer: 2/7

\newpage

\subsection*{Part 2}
Then you consider the following alternative implementation of the
same algorithm:

{\small
\begin{verbatim}
 
void generate_presents(){
 
  for (i=0; i<8; i++){
      for (j=0; j<8; j++) {
 
          if(db[i][j].age < 30)
             if(db[i][j].male) {
                db[i][j].nsync=0;
                db[i][j].britney_spears=1;
                db[i][j].garth_brooks=0;
                db[i][j].dolly_parton=0;
             }
                                                       
             else  
                db[i][j].nsync=1;
                db[i][j].britney_spears=0;
                db[i][j].garth_brooks=0;
                db[i][j].dolly_parton=0;                     
             }
          }                                             

          else{
             if(db[i][j].male) {
                db[i][j].nsync=0;
                db[i][j].britney_spears=0;
                db[i][j].garth_brooks=0;
                db[i][j].dolly_parton=1;
             }
 
             else{
                db[i][j].nsync=0;
                db[i][j].britney_spears=0;
                db[i][j].garth_brooks=1;
                db[i][j].dolly_parton=0;
             }
          }                                           


      }
   }
 
}
\end{verbatim}}                     

Making the same assumptions as in Part 1, answer the following questions.

A. What is the total number of read and write accesses? \verb|_______| 

% Answer: 8*8*6
 
B. What is the total number of read and write accesses that
miss in the cache? \verb|_______| 

% Answer: 8*4
 
C. So the fraction of all accesses that miss in the cache is: \verb|_______|.

% Answer: 1/12
                                                                   


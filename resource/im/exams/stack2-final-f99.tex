%%COMMENT:3:8:Stack discipline (buffer overflow) (Part II):
The following problem concerns the following, low-quality code:
\small{\begin{verbatim}
void foo(int x)
{
  int a[3];
  char buf[4];
  a[0] = 0xF0F1F2F3;
  a[1] = x;
  gets(buf);
  printf("a[0] = 0x%x, a[1] = 0x%x, buf = %s\n", a[0], a[1], buf);
}
\end{verbatim}}

In a program containing
this code, procedure {\tt foo} has the following disassembled form
on an IA32 machine:

\small{\begin{verbatim}
 080485d0 <foo>:
 80485d0:	55             	pushl  %ebp
 80485d1:	89 e5          	movl   %esp,%ebp
 80485d3:	83 ec 10       	subl   $0x10,%esp
 80485d6:	53             	pushl  %ebx
 80485d7:	8b 45 08       	movl   0x8(%ebp),%eax
 80485da:	c7 45 f4 f3 f2 	movl   $0xf0f1f2f3,0xfffffff4(%ebp)
 80485df:	f1 f0 
 80485e1:	89 45 f8       	movl   %eax,0xfffffff8(%ebp)
 80485e4:	8d 5d f0       	leal   0xfffffff0(%ebp),%ebx
 80485e7:	53             	pushl  %ebx
 80485e8:	e8 b7 fe ff ff 	call   80484a4 <_init+0x54>  # gets
 80485ed:	53             	pushl  %ebx
 80485ee:	8b 45 f8       	movl   0xfffffff8(%ebp),%eax
 80485f1:	50             	pushl  %eax
 80485f2:	8b 45 f4       	movl   0xfffffff4(%ebp),%eax
 80485f5:	50             	pushl  %eax
 80485f6:	68 ec 90 04 08 	pushl  $0x80490ec
 80485fb:	e8 94 fe ff ff 	call   8048494 <_init+0x44>  # printf
 8048600:	8b 5d ec       	movl   0xffffffec(%ebp),%ebx
 8048603:	89 ec          	movl   %ebp,%esp
 8048605:	5d             	popl   %ebp
 8048606:	c3             	ret    
 8048607:	90             	nop    
\end{verbatim}}

For the following questions, recall that:
\begin{denseitemize}
\item {\tt gets} is a standard C library routine. 
\item IA32 machines are little-endian.
\item C strings are null-terminated 
(i.e., terminated by a character with value 0x00). 
\item Characters `{\tt 0}' through `{\tt 9}' have ASCII codes
{\tt 0x30} through {\tt 0x39}.
\end{denseitemize}

\newpage

\begin{problem}{8}
Consider the case where procedure {\tt foo} is called with argument {\tt x}
equal to {\tt 0xE3E2E1E0}, and we type ``{\tt 123456789}'' in response
to {\tt gets}.  

\begin{choice}
\item Fill in the following table indicating which program values
are/are not corrupted by the response from {\tt gets}, i.e., their
values were altered by some action within the call to {\tt gets}.
\begin{center}
\renewcommand{\arraystretch}{1.5}
\begin{tabular}{|l|l|}
\hline
Program Value & Corrupted? (Y/N) \\
\hline
\verb@a[0]@ & \\
% Answer: Y
\hline
\verb@a[1]@ & \\
% Answer:  Y
\hline
\verb@a[2]@ & \\
% Answer:  N
\hline
\verb@x@ & \\
% Answer:  N
\hline
Saved value of register \verb@%ebp@ & \\
% Answer:  N
\hline
% Answer:  N
Saved value of register \verb@%ebx@ & \\
\hline
\end{tabular}
\end{center}

\item What will the {\tt printf} function print for the following:
\begin{itemize}
\item {\tt a[0]} (hexadecimal): \verb@________________________@
% Answer:  0x38373635
\item {\tt a[1]} (hexadecimal): \verb@________________________@
% Answer:  0xE3E20039
\item {\tt buf} (ASCII): \verb@________________________@
% Answer:  123456789
\end{itemize}
\end{choice}
\end{problem}



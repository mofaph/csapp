%%COMMENT:3:8:Decompiling a C ``for'' loop:
\begin{problem}{8}

Consider the following assembly representation of a function
{\tt foo} containing a {\tt for} loop:

\begin{scode}
\begin{alltt}
foo:{\em\scriptsize }
  pushl %ebp{\em\scriptsize }
  movl %esp,%ebp{\em\scriptsize }
  pushl %ebx{\em\scriptsize }
  movl 8(%ebp),%ebx{\em\scriptsize }
  leal 2(%ebx),%edx{\em\scriptsize }
  xorl %ecx,%ecx{\em\scriptsize }
  cmpl %ebx,%ecx{\em\scriptsize }
  jge .L4{\em\scriptsize }
.L6:{\em\scriptsize }
  leal 5(%ecx,%edx),%edx{\em\scriptsize }
  leal 3(%ecx),%eax{\em\scriptsize }
  imull %eax,%edx{\em\scriptsize }
  incl %ecx{\em\scriptsize }
  cmpl %ebx,%ecx{\em\scriptsize }
  jl .L6{\em\scriptsize }
.L4:{\em\scriptsize }
  movl %edx,%eax{\em\scriptsize }
  popl %ebx{\em\scriptsize }
  movl %ebp,%esp{\em\scriptsize }
  popl %ebp{\em\scriptsize }
  ret{\em\scriptsize }
\end{alltt}
\end{scode}

Fill in the blanks to provide the functionality of the loop:

{\small
\begin{alltt}
int foo(int a)
\{
    int i;
    int result = _____________;

    for( ________; ________; i++ ) \{

        __________________;

        __________________;

    \}
    return result;
\}
\end{alltt}
}

\answer{
int foo(int a)
{
    int i;
    int result = a + 2;

    for (i=0; i < a; i++) {
	result += (i + 5);
	result *= (i + 3);
    }
    return result;
}
}
\end{problem}

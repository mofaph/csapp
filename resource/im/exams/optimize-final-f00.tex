%%COMMENT:5:9:Program optimization:
\begin{problem}{9}

The following problem concerns optimizing a procedure for maximum
performance on an Intel Pentium III.  Recall the following performance
characteristics of the functional units for this machine:
\begin{center}
\begin{tabular}{|l|r|r|}
\hline
Operation & Latency & Issue Time \\
\hline
Integer Add & 1 & 1 \\
Integer Multiply & 4 & 1 \\
Integer Divide & 36 & 36 \\
Floating Point Add & 3 & 1 \\
Floating Point Multiply & 5 & 2 \\
Floating Point Divide & 38 & 38 \\
Load or Store (Cache Hit) & 1 & 1 \\
\hline
\end{tabular}
\end{center}

Consider the following two procedures:
\begin{center}
\begin{tabular}{|l|l|}
\hline
Loop 1	& Loop 2 \\
\hline
\verb@int loop1(int *a, int x, int n)@ 	&  \verb@int loop2(int *a, int x, int n)@ \\
\verb@{@ 				&  \verb@{@ \\
\verb@  int y = x*x;@			&  \verb@  int y = x*x;@ \\
\verb@  int i;@				&  \verb@  int i;@ \\
\verb@  for (i = 0; i < n; i++)@	&  \verb@  for (i = 0; i < n; i++)@ \\
\verb@    x = y * a[i];@		&  \verb@    x = x * a[i];@ \\
\verb@  return x*y;@			&  \verb@  return x*y;@ \\
\verb@}@				&  \verb@}@ \\
\hline
\end{tabular}
\end{center}


When compiled with GCC, we obtain the following assembly code for the inner loop:
\begin{center}
\begin{tabular}{|l|l|}
\hline
Loop 1	& Loop 2 \\
\hline
\verb@.L21:@				&	\verb@.L27:@ \\
\verb@   movl %ecx,%eax@		&	\verb@   imull (%esi,%edx,4),%eax@ \\
\verb@   imull (%esi,%edx,4),%eax@	&	\verb@   incl %edx@ \\
\verb@   incl %edx@			&	\verb@   cmpl %ebx,%edx@ \\
\verb@   cmpl %ebx,%edx@		&	\verb@   jl .L27@ \\
\verb@   jl .L21@			& \\
\hline
\end{tabular}
\end{center}


Running on one of the Fish machines, we find that Loop 1 requires
3.0 clock cycles per iteration, while Loop 2 requires 4.0.

\begin{choice}
\item
Explain how it is that Loop 1 is faster than Loop 2, even though it
has one more instruction
%% Answer: Loop1 does not have any loop-carried dependency.  It can therefore
%% make better use of pipelining.
\shortanswer{0.5}
\item
By using the compiler flag {\tt -funroll-loops}, we can compile the
code to use 4-way loop unrolling.  This speeds up Loop 1.  Explain why.
%% Answer: It reduces the overhead for each iteration.
\shortanswer{0.5}
\item
Even with loop unrolling, we find the performance of Loop 2 remains the same.  Explain why.
%% Answer: Still limited by latency of integer multiply (4 cycles).
\shortanswer{0.5}
\end{choice}


\end{problem}
%%COMMENT:2:9:Two's complement encoding and arithmetic:
\begin{problem}{9}

Assume we are running code on a $6$-bit machine using two's complement
arithmetic for signed integers. A ``short'' integer is encoded using
$3$ bits. Fill in the empty boxes in the table below. The following
definitions are used in the table:

\begin{verbatim}
short sy = -3;
int y = sy;
int x = -17;
unsigned ux = x;
\end{verbatim}

Note: You need not fill in entries marked with ``--''.

\begin{center}
\renewcommand{\arraystretch}{1.8}
\begin{tabular}{|c|c|c|} \hline 
\makebox[1.5in]{Expression} & \makebox[1.5in]{Decimal Representation} &
 \makebox[1.5in]{Binary Representation} \\ \hline \hline

Zero & $0$ & \\ \hline

-- & $-6$ &  \\ \hline

%-- & $29$ & \\ \hline

-- & & {\tt 01 0010} \\ \hline

$ux$ & & \\ \hline

$y$ & & \\ \hline

$x >> 1$ & &  \\ \hline

TMax & & \\ \hline

$-$TMin & & \\ \hline

%TMax $+$ TMax & & \\ \hline

TMin $+$ TMin & & \\ \hline

%TMax $+$ TMin & & \\ \hline


\end{tabular} 
\end{center}

\end{problem}

\comment{
 Expression   decimal     binary
-----------------------------------
Zero       |     0     |  00 0000           
---        |    -6     |  11 1010
---        |    18     |  01 0010
ux         |    47     |  10 1111
y          |    -3     |  11 1101
x >> 1     |    -9     |  11 0111
TMax       |    31     |  01 1111
-TMin      |   -32     |  10 0000
TMin+TMin  |     0     |  00 0000
-----------------------------------
}


%%COMMENT:5:10:Associativity and its relationship to parallelism (cool!):
\begin{problem}{10}
\end{problem}

Consider the following function for computing the product of an array
of $n$ integers.  We have unrolled the loop by a factor of 3.

\begin{ccode}
\begin{alltt}
int aprod(int a[], int n)
\verb:{:
    int i, x, y, z;
    int r = 1;
    for (i = 0; i < n-2; i+= 3) \verb:{:
        x = a[i]; y = a[i+1]; z = a[i+2];
        r = r * x * y * z; // Product computation
    \verb:}:
    for (; i < n; i++)
        r *= a[i];
    return r;
\verb:}:
\end{alltt}
\end{ccode}

For the line labeled {\tt Product computation}, we can use parentheses
to create 5 different associations of the computation, as follows:

\begin{ccode}
\begin{alltt}
        r = ((r * x) * y) * z; // A1
        r = (r * (x * y)) * z; // A2 
        r = r * ((x * y) * z); // A3
        r = r * (x * (y * z)); // A4
        r = (r * x) * (y * z); // A5
\end{alltt}
\end{ccode}

We express the performance of the function in terms of the number of
cycles per element (CPE).  As described in the book, this measure
assumes the run time, measured in clock cycles, for an array of length
$n$ is a function of the form $C n + K$, where $C$ is the CPE\@.

We measured the 5 versions of the function on an Intel Pentium III\@.
Recall that the integer multiplication
operation on this machine has a latency of 4 cycles and an issue time
of 1 cycle.

\newpage
(continued)

The following table shows some values of the CPE, and other values
missing.  The measured CPE values are those that were actually
observed.  ``Theoretical CPE'' means that performance that would be
achieved if the only limiting factor were the latency and issue time
of the integer multiplier.

\begin{center}
\renewcommand{\arraystretch}{1.3}
\begin{tabular}{|c|c|c|}
\hline
\makebox[1in]{Version} & \makebox[2in]{Measured CPE} & \makebox[2in]{Theoretical CPE} \\
\hline
{\tt A1} & 4.00 &\\ 		
\hline
{\tt A2} & 2.67 &\\ 		
\hline
{\tt A3} & & $4/3 = 1.33$\\     
\hline
{\tt A4} & 1.67 &\\ 		
\hline
{\tt A5} &&$8/3 = 2.67$\\       
\hline
\end{tabular}
\end{center}

Fill in the missing entries.  For the missing values of the measured
CPE, you can use the values from other versions that would have the
same computational behavior.  For the values of the theoretical CPE,
you can determine the number of cycles that would be required for an
iteration considering only the latency and issue time of the
multiplier, and then divide by 3.

%
%Answers:
%
%Version Measured CPE	Theoretical CPE
%A1	 4.00		4/1 = 4.00
%A2	 2.67		8/3 = 2.67
%A3	 1.67		4/3 = 1.33
%A4	 1.67		4/3 = 1.33
%A5	 2.67		8/3 = 2.67
%